\documentclass[12pt,a4paper,twoside]{article}
\usepackage{geometry}
 \geometry{
 a4paper,
 left=20mm,
 top=0mm,
 bottom=45mm,
 headheight=30mm,
 heightrounded,
 includehead,
 includefoot
 }
\usepackage{graphicx}
\usepackage{amsfonts, amssymb, amsmath}
\usepackage[headings]{fancyhdr}
\usepackage{helvet}
\renewcommand{\rmdefault}{\sfdefault}
\pagestyle{fancy}
\fancyhead[R]{\includegraphics[width=3cm]{images/atec.png}}
\fancyhead[L]{\includegraphics[width=2cm]{images/taubatexas.png}}
\fancyfoot[L]{\scriptsize ATEC\\Associação Taubatexas para o Esporte e Conhecimento\\Telefone: (12) 99261-9898 / CNPJ: 37.124.329.0001-74}
\fancyfoot[R]{\scriptsize E.E. Engenheiro Urbano Alves de Souza Pereira\\R. Antônio Rodrigues Miranda, 170\\CEP: 12031-580 - Taubaté - SP}
\setlength{\footskip}{80pt}
\setlength{\arrayrulewidth}{0.5mm}
\setlength{\tabcolsep}{18pt}
\renewcommand{\arraystretch}{1.5}

\begin{document}
\section{Introduction}
The distributive property states that $a(b + c) = ab + ac$, for all $a, b, c \in \mathbb{R}$.\par
The well known Pythagorean theorem \(x^2 + y^2 = z^2\) was 
proved to be invalid for other exponents. 
Meaning the next equation has no integer solutions:
\[ x^n + y^n = z^n \]
\begin{tabular}{ |p{3cm}|p{3cm}|p{3cm}|  }
\hline
\multicolumn{3}{|c|}{Team List} \\
\hline
Name & City & State/Province \\
\hline
254 & San Jose & CA \\
1678 & Davis & CA \\
2056 & Stoney Creek & ON \\
2910 & Mill Creek & WA \\
1323 & Madera & CA \\
148 & Greenville & TX   \\
27 & Clarkston & MI \\
\hline
\end{tabular}
\end{document}
